\chapter{Introduction}
\label{chapter:intro}

 In modern era we have seen phenomenal increase in human dependency on information and communication technology. ICT enabled products and services has transformed the way of life on the planet. We need and depend on ICT to fulfil our needs from basic physiological level to the human desire of being effective part of society. There are many research areas and opportunities that are emerging as bi-products of this continuous transformation. One of them is the availability of digital traces of human activities. Every time we use these services, we produce digital traces that can be recorded and analysed. Big Data is a term that is being widely used to refer to these digital traces of human activity. Ubiquity of computing resources, fast and highly mobile connectivity and advent of social media usage has caused a great surge in volumes of data. Realizing the true potentials of data, businesses are not only utilizing it as source of decision making but as a new revenue stream. Large scale opportunities are emerging that are reshaping the business models of many companies around the globe.
 
To support this transfiguration, we have seen a rapid development in distributed parallel computing, data communication software and machine learning. Industry giants like Google and Yahoo has opened technologies and tools like MapReduce and Hadoop to facilitate these advancement and open source software communities like Apache Software foundation has further developed the tools to provide a complete ecosystem for handling big data and generate insights. The new specialized big data companies like Cloudera and Hortonworks has emerged as catalyst for this data revolution. In this research we try to formulate a model for end to end big data analytics platform based on these technologies that can ingest data from heterogeneous sources, process it in an efficient way, mine the data to generate the insights based on business logic and then present the information using interactive visualizations. This thesis includes the development as well as implementation of the mentioned big data platform to perform analysis on real life use cases and generate useful insights. The model that we present in this thesis is based on open source software components available free of charge. There are other closed source software alternatives that can fit into the presented model but they are not discussed in this scope of this thesis.

This thesis is inspired by European Union CIVIS- Cities as drivers of social change project under 7th framework. CIVIS project focuses on adoption of ICT tools and techniques for integrating social aspects of city life into production, distribution and consumption of energy. It aims to make city life as functional unit to achieve global goal of low carbon emissions from energy ecosystem. The use of pervasive ubiquitous computing is driving the smart energy solutions. Combined with internet of things (IoT) for home/building automation, smart commuting, and remote monitoring is becoming the basis for energy conservation via gaining energy efficiency. All the smart energy devices as part of this ecosystems generates high volumes of data, that needs to be instantaneously transferred, stored, analysed and visualized for knowledge discovery and  improvements of services for the goal of achieving high energy efficiency. The platform that was developed as part of this thesis has the capability to automate the whole process.
 
Energy usage pattern detection, classification of buildings on basis of energy efficiency and a prediction model for energy consumption per household will be the use cases for validating the developed big data analytics platform. These use cases also provide the basis for designing, planning and implementing schemes for improving energy related services for sake of achieving higher efficiency in both production and usage that contributes to cause of greener environment. The insight generated from these use cases can also help in educating the consumer about benefits of energy conservation and spread the awareness about behavioural changes that can benefit society as well as individuals. 

This master thesis is also supported by VTT, Technical Research Centre of Finland as part of their Green Campus initiative that focuses on use of ICT based solutions for innovative energy management and control systems capable to optimize the consumption without compromising the indoor environment. VTT is also a supporter and partner of CIVIS project. VTT has installed specialized smart devices in selected test sites that are the buildings owned by Aalto University. VTT has contributed to this thesis by providing the data generated by these smart devices. VTT has also helped in scoping for the use cases for energy efficiency by the experience and the knowledge they have from the related projects and research.

In a nutshell, this thesis focuses on providing a solution for collecting, storing, analysing and visualizing data generated by smart energy device for generating insights about energy consumption patterns and discovering the performance of different building units in terms of energy efficiency. This thesis also provides the models for knowledge discovery that can be used to improve energy efficiency at both producers and consumers ends. The big data analytics platform developed as part of this thesis is not limited to be used only for energy efficiency. It has the capability of handling other big data uses cases as well. However, within scope of this document we shall only discuss its use for energy usage patterns detection and efficiency.     





\section{Problem statement}

Energy conservation is required to reduce C02 emissions from energy production and usage. To achieve this goal we need to understand and improve the energy efficiency on both producer and consumer end. ICT enabled smart energy grids and devices are being rolled out globally to measure energy consumption and improve energy efficiency. These smart devices produce high volumes of data that may or may not be predicted and planned at time of setting up the infrastructure. The data generated by different devices comes in different formats. For knowledge discovery from this data it is required to collect, store and analyse the data and then visualize the generated insights in a way that the information can be understood efficiently. The challenge gets even tougher when data needs to be collected and analysed in real time. Then with the time, volume of data and scope of analysis is expected to increase. So to cater for all this a highly scalable and flexible data analysis platform is required that can automate the whole process. This platform needs to be very cost effective for global adaptation.
 
In scope of this research we provide a model for big data analytics platform that can provide the solution for these requirements. We also implement the proposed model and test it with real life data from smart energy devices. The proposed solution is based on open source components that can be deployed on general purpose hardware that can be procured very easily and inexpensively. The proposed platform can be scaled according to data requirements and additional functional components can be integrated as per the scope of analysis.


\section{Helpful hints}

For referencing we shall be using Vancouver system\cite{neville2012referencing}. When discussing from authors point of view we shall be using Author\((s)\) name, year of publication as our format along with sequence numbering from Vancouver systm. In case of quotation from author we shall be using double quotes e.g. ``quotation as it is''.

Throughout the document we shall be discussing about energy. Due to our main focus, The term energy in our research  and this document will refer to electricity or electric power. In case of all other types of energy we shall be specifically mentioning the type name along with energy as a term.

In this document we shall be discussing about the concept, development and use of a big data platform as our main environment for our data analysis. The terms like platform, data platform, and big data platform will be used to refer to the same concept. In case of a specific need of any other platform like concept we shall be giving proper descriptions.


\section{Structure of the Thesis}
\label{section:structure} 

to be written at document finalization stage. 

